%%
%% The following code sets up the document formatting
%%

%this assumes that res_yy.sty is in some path
\documentstyle[hyperref, margin, line]{res_yy}


\hypersetup{backref,pdfpagemode=Full,colorlinks=true,backref}

\addtolength{\oddsidemargin}{-0.45in}
\addtolength{\voffset}{-0.30in}
\addtolength{\textwidth}{1.00in} \addtolength{\textheight}{1.50in}

\renewcommand{\namefont}{\LARGE\emph}



%%
%% The following code defines some macros for terms which have raised font
%% (ie 4\fourth would result 4th with the 'th' raised (superscripted)
%%

\def\Cplusplus{{C\raise.5ex\hbox{\small ++}}}
\def\CSharp{{C\raise.5ex\hbox{\small \#}}}
% 'st' 'nd' 'rd' 'th' superscripts for numbers
\def\first{{\raise.5ex\hbox{\small st}}}
\def\second{{\raise.5ex\hbox{\small nd}}}
\def\third{{\raise.5ex\hbox{\small rd}}}
\def\fourth{{\raise.5ex\hbox{\small th}}}

%%
%% starting the actual document
%%

%\usepackage{fontspec}
%\usepackage{color}
%\usepackage[usenames,dvipsnames]{color}

\begin{document}
%\setmainfont{Adobe Garamond Pro}

\setlength{\parindent}{0pt}
\setlength{\hangindent}{0pt}
\setlength{\leftskip}{0pt}
\setlength{\parskip}{0pt}

%\newcommand{\tt}[1]{\texttt{#1}}

\definecolor{dark-blue}{rgb}{0,0,0.12}
\newcommand{\blue}[1]{\textcolor{dark-blue}{#1}}
\definecolor{brass}{rgb}{0.2,0.2,0}
\newcommand{\brass}[1]{\textcolor{brass}{#1}}
\newcommand{\headstyle}[1]{\blue{#1}}

\newcommand{\FIXME}[1]{\textcolor{red}{\emph{FIXME #1}}}
\newcommand{\kw}[1]{\blue{#1}}
\newcommand{\heavy}[1]{\textbf{#1}}

\name{\brass{Zachary McCord}}

\address{
\brass{
{\tt zjmccord@gmail.com}\\
3336 Dawson St\\
Pittsburgh, PA 15213\\
847.754.7987
}
}

\begin{resume}
\end{resume}


\section{\brass{\textsc{Education}}}

\headstyle{
\textbf{Carnegie Mellon University} \hfill August 2006 - December 2010
}

B.S. in Computer Science with a minor in Mathematical Sciences. GPA of
$3.\bar0$.

%%
%% the meat of the resume starts now
%%

\begin{formatb}
\employer{l}\title{r}\\
\location{l}\dates{r}\\
\body\\
\end{formatb}

%\section{\brass{\textsc{Objective}}}
%
%I'm motivated most by learning and novelty and work best in a small to medium group. Since
%I'm still finding my interests and talents I'm looking particularly for a position where the work
%is both varied and challenging. 

\section{\brass{\textsc{Skills}}}
\begin{description}

\item[\headstyle{Systems}]
Comfortable developing on \kw{Linux}, \kw{Mac OS X}, \kw{Windows} and \kw{Atmel AVR-8}.
\item[\headstyle{Languages}]
Proficient in \kw{C}, \kw{\Cplusplus} and \kw{Perl};
also know \kw{Qt}, \kw{\CSharp}, \kw{Java}, \kw{bash}, \kw{Objective-C}, \kw{Verilog}, \kw{Ruby},
\kw{\LaTeX}, \kw{SML}, \kw{OCaml}, \kw{Lisp}, \kw{Scheme} and \kw{OpenGL}.
%\item[Tools]
%Comfortable with \kw{doxygen}, \kw{mercurial}, \kw{svn}
\item[\headstyle{Work style}]
I comment thoroughly and work best in small groups.


\end{description}

\section{\brass{\textsc{Experience}}}

\headstyle{
\textbf{Pittsburgh Pattern Recognition, Inc. (now part of Google)} \hfill Intern \\
Pittsburgh, PA                                \hfill April to July 2011
}
\begin{itemize}
\item Wrote the first live, interactive face recognition and tracking demo for
      PittPatt's latest SDK release.
\item Integrated demo with SDK prior to SDK release to debug the
      build-release-and-support chain.
\item Caught several product bugs before they affected customers.
\end{itemize}
\setlength{\parskip}{8pt}

\headstyle{
\textbf{School of Computer Science, Carnegie Mellon University} \hfill Teaching Assistant \\
Pittsburgh, PA                                                  \hfill January to May 2011
}
\setlength{\parskip}{0pt}
\begin{itemize}
\item Created assignments and built course infrastructure for 15-495, \kw{Information Forensics}.
\end{itemize}
\setlength{\parskip}{8pt}

\headstyle{
\textbf{Language Technologies Institute, Carnegie Mellon University} \hfill Student Researcher\\
Pittsburgh, PA \hfill October 2008 to August 2009
}
\setlength{\parskip}{0pt}

Student researcher under Professor Noah Smith (see \headstyle{References}).

%I performed open-ended research and supported other researchers within the ARK
%research group:
\begin{itemize}
\item Managed and processed data on
      our 20-node \kw{Condor} computing grid.
\item Researched handwritten grammars for quote
      extraction, obtaining significant performance gains immediately
      useful to others in the research group.
\end{itemize}
\setlength{\parskip}{8pt}

\headstyle{
\textbf{Robotics Institute, Carnegie Mellon University} \hfill  Student Researcher \\
Pittsburgh, PA                                           \hfill  September 2006 to August 2007
}
\setlength{\parskip}{0pt}
\begin{itemize}
\item Developed 3D robot visualization software in \kw{C} and \kw{OpenGL}.
\item Wrote \kw{C} code for \kw{AVR-8} microcontroller to control a modular robot over a serial bus.
\item Worked on redesign and rewrite of Windows/Linux robot control software.
%\item Soldered a \emph{lot} of wires.
\end{itemize}


\section{\brass{\textsc{Coursework}}}
\headstyle{
\textbf{Operating Systems Design and Implementation}
}

My partner and I wrote a thread library in \kw{C}, and wrote the first \kw{\Cplusplus}\ 
kernel
ever submitted in the class.

\headstyle{
\textbf{Introduction to Computer Architecture}
}

Our team of 3 designed a simple, pipelined MIPS-like processor core in
\kw{Verilog}.

\headstyle{
\textbf{Introduction to Natural Language Processing}
}

Our team of 3 wrote a question answering system based on Wikipedia articles,
using \kw{SML} and \kw{Perl}.


%%
%% We use the same formatting for projects as for work experience
%% Shown below is the formatting used previously
%%
%%  \begin{formatb}
%%    \employer{l}\title{r}\\
%%    \location{l}\dates{r}\\
%%    \body\\
%%  \end{formatb}
%%
%% 
%%  Note that \location is now being used for non-location information
%%


\begin{formatb}
  \employer{l}\dates{r}\\
  \body\\
\end{formatb}

\section{\brass{\textsc{Publications}}}

\headstyle{
\textbf{Design of a Modular Snake Robot} \hfill IROS 2007
}

Cornell Wright, Aaron Johnson, Aaron Peck, Zachary McCord, Allison Naaktgeboren, Philip Gianfortoni, Manuel Gonzalez-Rivero, Ross Hatton, and Howie Choset

\section{\brass{\textsc{References}}}

\begin{description}
\item[\headstyle{Michael A. Sipe, Ph.D.}] VP of Product Development, Pittsburgh
Pattern Recognition, Inc.\\
\texttt{sipe@ieee.org}. 
\item[\headstyle{Brian R. Colonna}] Software Engineer, Pittsburgh Pattern
Recognition, Inc.\\
\texttt{brian.r.colonna@gmail.com}. 
\item[\headstyle{Prof. Noah Smith}] Associate Professor, Language Technologies Institute, Carnegie
Mellon University. \texttt{nasmith@cs.cmu.edu}
%\item Andrew Drake. \texttt{adrake@andrew.cmu.edu}. My Operating Systems partner.
\end{description}

\end{document}
