%%%%%%%%%%%%%%%%%%%%%%%%%%%%%%%%%%%%%%%%%%%%%%%%%%%%%%%%%%%%%%%%%%%
%% 
%% Yisong Yue's resume
%%   - based off work by Michael DeCorte 
%%
%%%%%%%%%%%%%%%%%%%%%%%%%%%%%%%%%%%%%%%%%%%%%%%%%%%%%%%%%%%%%%%%%%%



%%
%% The following code sets up the document formatting
%%

%this assumes that res_yy.sty is in some path
\documentstyle[hyperref, margin, line]{res_yy}

\hypersetup{backref,pdfpagemode=Full,colorlinks=true,backref}

\addtolength{\oddsidemargin}{-0.45in}
\addtolength{\voffset}{-0.30in}
\addtolength{\textwidth}{1.00in} \addtolength{\textheight}{1.50in}

\renewcommand{\namefont}{\LARGE\emph}



%%
%% The following code defines some macros for terms which have raised font
%% (ie 4\fourth would result 4th with the 'th' raised (superscripted)
%%

\def\Cplusplus{{\rm C\raise.5ex\hbox{\small ++}}}
\def\CSharp{{\rm C\raise.5ex\hbox{\small \#}}}
% 'st' 'nd' 'rd' 'th' superscripts for numbers
\def\first{{\raise.5ex\hbox{\small st}}}
\def\second{{\raise.5ex\hbox{\small nd}}}
\def\third{{\raise.5ex\hbox{\small rd}}}
\def\fourth{{\raise.5ex\hbox{\small th}}}



%%
%% starting the actual document
%%

\begin{document}

%the name in big fonts at the top of resume
%this is left aligned
\name{Yisong Yue}

%this is right aligned
\address{
website: www.yisongyue.com \ \ \ \ \ email: yisongyue@gmail.com  
}

\begin{resume}



%%
%% This section of code is inelegant, but I'm too lazy to fix it
%%

\section{\textsc{Research Interest}}
My research interests lie primarily in machine learning and its applications.

\section{\textsc{Education}}

\textbf{Cornell University} \hfill 2005 - Present \\
PhD Student in Computer Science \hfill Advisor: Thorsten Joachims \\
\newline
\textbf{University of Illinois at Urbana-Champaign} \hfill 2001 - 2005 \\ 
Bachelor of Science in Computer Science \hfill Graduated with Highest Honors



%%
%% the meat of the resume starts now
%%

\begin{formatb}
  \employer{l}\title{r}\\
  \location{l}\dates{r}\\
  \body\\
\end{formatb}

\section{\textsc{Work Experience}}

\employer{\textbf{NVIDIA Corporation}}
\title{Architecture Engineer Intern}
\location{Santa Clara, CA}
\dates{Summer 2005}
\begin{position}
Developed internal tools to assist GPU simulation and data mining.
\end{position}

\employer{\textbf{Microsoft Corporation}}
\title{Software Design Engineer Intern}
\location{Redmond, WA}
\dates{Summer 2004}
\begin{position}
Designed and wrote a customized certificate API for authentication and data encryption.
\end{position}

\employer{\textbf{Microsoft Corporation}}
\title{Software Design Engineer Intern}
\location{Redmond, WA}
\dates{Summer 2003}
\begin{position}
Used DirectX with HLSL to prototype imaging effects that are
processed on the GPU.
\end{position}


%%
%% We use the same formatting for projects as for work experience
%% Shown below is the formatting used previously
%%
%%  \begin{formatb}
%%    \employer{l}\title{r}\\
%%    \location{l}\dates{r}\\
%%    \body\\
%%  \end{formatb}
%%
%% 
%%  Note that \location is now being used for non-location information
%%


\begin{formatb}
  \employer{l}\dates{r}\\
  \body\\
\end{formatb}

\section{\textsc{Publications}}

\employer{\textbf{A Support Vector Method for Optimizing Average Precision}}
\dates{SIGIR 2007}
\begin{position}
Authors: Yisong Yue, Thomas Finley, Filip Radlinski, Thorsten Joachims
\end{position}

\section{\textsc{Other Projects}}

\employer{\textbf{Finding Influential Blogs via Link Prediction}}
\dates{Spring 2006}
\begin{position}
Used machine learning and link analysis techniques to determine the amount of influence blogs exert on each other.
\end{position}

\employer{\textbf{Loss-Minimizing Voting for Machine Learning Ensembles}}
\dates{Spring 2006}
\begin{position}
Explored voting schemes which minimizes a loss function for an ensemble of learning models. 
\end{position}

\employer{\textbf{Parameter Estimation for MRF-Stereo with Occlusions}}
\dates{Fall 2005}
\begin{position}
Used an EM-method to iteratively compute superior parameters for the baseline MRF-stereo algorithm with occlusions.
\end{position}

\employer{\textbf{Illini Book Exchange}}
\dates{2002-2005}
\begin{position}
http://www.illinibookexchange.com\\
Worked on development, management and marketing of Illini Book Exchange for the Technology and Management Club at UIUC. 
\end{position}

\employer{\textbf{Reflections Projections}}
\dates{Fall 2004}
\begin{position}
http://www.acm.uiuc.edu/conference/\\
Helped plan and manage Reflections Projections 2004 as Treasurer of ACM @ UIUC.
\end{position}


%%
%% This section could also use more formatting, but looks ok, as is
%%

%\section{\textsc{Qualifications}}

%\emph{Programming Languages}: \Cplusplus, \CSharp, Cg, HLSL, ARB assembly, SML, OCaML, PHP, MySQL, Java, Python, Perl, MIPS assembly

%\emph{Libraries and Tools}: Vim, STL, DirectX, OpenGL, \LaTeX, GIMP, Adobe Suite, Macromedia Suite, MatLab, Mathematica, Microsoft Visual Studio, GCC, GDB


%%
%% Note that we're redefining the formatting
%% We only have one row of information now, instead of two
%%

\section{\textsc{Activities}}

\begin{formatb}
  \employer{l}\dates{r}\\
  \body\\
\end{formatb}

\employer{\textbf{Cornell Teaching Assistant (TA Excellence Award)}}
\dates{Fall 2005 - Spring 2006}
\begin{position}
Taught two sections of CS 100M during the Fall and Spring semesters of the 2005-6 academic year.  Received award in recognition of performance.
\end{position}

\employer{\textbf{UIUC ACM Treasurer}}
\dates{Fall 2004 - Spring 2005}
\begin{position}
Managed all financial responsibilities of local chapter of ACM.  Assisted the Chair in general management of ACM.
\end{position}

\employer{\textbf{UIUC ACM SIGGRAPH Chair}}
\dates{Spring 2004}
\begin{position}
Managed the local chapter of SIGGRAPH, organized projects and workshops/tutorials
\end{position}



%%
%% Nothing special here, just a normal table
%%

%\section{\textsc{Course Work}}
%  \begin{tabular}{lllll}
%  Information Networks   & \ \ & Machine Learning    & \ \ & Theory of Computation \\ 
%  Computer Graphics      & \ \ & Machine Vision      & \ \ & Programming Languages \\
%  Software Engineering   & \ \ & Algorithms          & \ \ & Artificial Intelligence     \\
%  Operating Systems      & \ \ & Databases           & \ \ & Computer Architecture \\
%  Numerical Methods      & \ \ & Graph Theory        & \ \ & Differential Equations      \\
%  Probability Theory     & \ \ & Number Theory       & \ \ & Differential Geometry       \\
%  Advanced Calculus      & \ \ & Abstract Algebra    & \ \ & Advanced Combinatorics   \\
%  \end{tabular}


\end{resume}
\end{document}
